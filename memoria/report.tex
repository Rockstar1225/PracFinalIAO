\documentclass[es]{uc3mthesisIEEE}


\usepackage{import}
\usepackage{enumitem}  % control item separation -> \begin{itemize}[nosep]
\usepackage{lipsum}  % dummy text
\usepackage{placeins}  % \FloatBarrier -> prevents figures and tables from passing that point

\usepackage{mymacros}  % report-specific macros


% silence ht warnings
\usepackage{silence}
\WarningFilter{latex}{`h' float specifier changed to `ht'}


% REFERENCES
\addbibresource{references.bib}  % bibliography file
\import{}{glossary.tex}  % glossary file


%	DOCUMENT

% setup
\degree{Ingeniería Informática}
\title{Detección de tumores cerebrales con redes de neuronas artificiales}
\shorttitle{Detección de tumores cerebrales}
\author{Tomás Mendizábal -- 100461170 \\
        Adrián Cortázar Lería -- 100545860 \\
        Ángel Pérez Navas -- 100472200 \\
        José Antonio Verde Jiménez -- 100472221}
\advisors{Elías Nezamoleslami}
\place{Leganés, Madrid, España}
\date{Diciembre 2024}

\begin{document}

  % COVER
  \makecover


  % EPIGRAPH
  \makeepigraph
    {Tiene que ser formato TFG}  % quote
    {Anónimo}  % author
    {}  % source

  % ABSTRACT
  \begin{abstract}
    \lipsum[1-3]
    \keywords{patata \sep papas \sep manzanas de tierra}
  \end{abstract}


  % TOC
  \tableofcontents
  \listoffigures
  \listoftables


  % THESIS
  \begin{thesis}
    % \includefrom{parts/}{introduction.tex}
  \end{thesis}


  % BIBLIOGRAPHY
  \cleardoublepage
  \label{bibliography}
  \printbibliography[heading=bibintoc]


  % GLOSSARY
  \cleardoublepage
  \label{glossary}
  \printglossaries
  % \printnoidxglossaries[type=\acronymtype]  % slower, but no need to do $ makeglossaries report


  % APPENDICES
  % \begin{appendices}
  %   \chapter{My stuff}
  %   \lipsum
  % \end{appendices}


\end{document}
